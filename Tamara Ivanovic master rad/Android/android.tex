\documentclass[../TamaraIvanovicMasterRad.tex]{subfiles}
\usepackage{subfiles}
\documentclass[12pt,oneside]{memoir} 
\usepackage[latinica]{matfmaster} 

% ------------------
\usepackage{listings}
\usepackage{xcolor}

\definecolor{codegreen}{rgb}{0,0.6,0}
\definecolor{codegray}{rgb}{0.5,0.5,0.5}
\definecolor{codepurple}{rgb}{0.58,0,0.82}
\definecolor{backcolour}{rgb}{0.95,0.95,0.92}

\lstdefinestyle{mystyle}{
    backgroundcolor=\color{backcolour},   
    commentstyle=\color{codegreen},
    keywordstyle=\color{magenta},
    numberstyle=\tiny\color{codegray},
    stringstyle=\color{codepurple},
    basicstyle=\ttfamily\footnotesize,
    breakatwhitespace=false,         
    breaklines=true,                 
    captionpos=b,                    
    keepspaces=true,                 
    numbers=left,                    
    numbersep=5pt,                  
    showspaces=false,                
    showstringspaces=false,
    showtabs=false,                  
    tabsize=2
}

\lstset{style=mystyle}
% ------------------
\begin{document}

%-------------------------------------------------------------------------------------------
Operativni sistem Android (u nastavku OS Android) je operativni sistem zasnovan na Linuks jezgru (eng. \textit{Linux kernel}) i pripada zajednici otvorenog koda. U ovom poglavlju biće reči o samom nastanku i razvoju ovog operativnog sistema, arhitekturi i osnovnim kompontentama. Kako je centralna tačka ovog rada aplikacija koja kontroliše set top-boks (eng. \textit{set top-box}, skraćeno STB) uređaje koja je pisana u programskom jeziku Java biće objašnjen i odnos OS-a Android sa njima. Radi boljeg razumevanja rada aplikacije ovo poglavlje će se osvrnuti i na značaj i funkcionisanje Google API-ja, kao i na metod povezivanja dva uređaja sa OS-om Android pomoću mreže. 
%------------------------------------------------------------------------------------------
\section{Istorijat}
\subfile{istorijat}

%-----------------------------------------------------------------------------------------

\section{Arhitektura}
\subfile{arhitektura}

%------------------------------------------------------------------------------------------

\section{Komponente Android aplikacije}
\subfile{komponente}

%------------------------------------------------------------------------------------------
\section{Android i STB}
STB uređaji su namenjeni za pružanje digitalnih televizijskih usluga korisnicima, a koristeći OS Android, ovi uređaji mogu da pruže mnogo više funkcionalnosti. Kako OS Android pripada zajednici otvorenog koda proizvođači STB uređaja mogu lako prilagoditi sistem svojim potrebama. Takođe moguće je koristiti \textit{Google} prodavnicu čime se broj aplikacija koje se mogu koristiti na uređajima znatno povećava. Pored ovoga moguće je pokretati svoje aplikacije koje će raditi samostalno ili u interakciji sa drugim instaliranim aplikacijama. Sigurnost aplikacija koje se kreiraju za STB uređaje je u stalnom porastu s obzirom da nove verzije OS Android donose sa sobom veću stabilnost i bezbednost, a nove verzije često izlaze. 

\section{Android i Java}
\section{Google API}
\subfile{googleApi}
%\section{Povezivanje Android uređaja preko mreže}

\end{document}

