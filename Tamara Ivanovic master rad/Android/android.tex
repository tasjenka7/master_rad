\documentclass[../TamaraIvanovicMasterRad.tex]{subfiles}
\usepackage{subfiles}
\documentclass[12pt,oneside]{memoir} 
\usepackage[latinica]{matfmaster} 

% ------------------
\usepackage{listings}
\usepackage{xcolor}

\definecolor{codegreen}{rgb}{0,0.6,0}
\definecolor{codegray}{rgb}{0.5,0.5,0.5}
\definecolor{codepurple}{rgb}{0.58,0,0.82}
\definecolor{backcolour}{rgb}{0.95,0.95,0.92}

\lstdefinestyle{mystyle}{
    backgroundcolor=\color{backcolour},   
    commentstyle=\color{codegreen},
    keywordstyle=\color{magenta},
    numberstyle=\tiny\color{codegray},
    stringstyle=\color{codepurple},
    basicstyle=\ttfamily\footnotesize,
    breakatwhitespace=false,         
    breaklines=true,                 
    captionpos=b,                    
    keepspaces=true,                 
    numbers=left,                    
    numbersep=5pt,                  
    showspaces=false,                
    showstringspaces=false,
    showtabs=false,                  
    tabsize=2
}

\lstset{style=mystyle}
% ------------------
\begin{document}

%-------------------------------------------------------------------------------------------
Android operativni sistem (u nastavku Android OS) je operativni sistem zasnovan na Linux jezgru i pripada zajednici otvorenog koda. U ovom poglavlju biće reči o samom nastanku i razvoju ovog operativnog sistema, arhitekturi i osnovnim kompontentama. Kako je centralna tačka ovog rada aplikacija koja kontroliše set top-box uređaje (STB) koja je pisana u programskom jeziku Java ovo poglavlje će se osvrnuti i na odnos Android OS-a sa njima. Radi boljeg razumevanja rada aplikacije ovo poglavlje će se osvrnuti i na značaj i funkcionisanje Google API-ja, kao i na metod povezivanja dva uređaja sa Android OS-om pomoću mreže. 
%------------------------------------------------------------------------------------------
\section{Istorijat}
\subfile{istorijat}

%-----------------------------------------------------------------------------------------

\section{Arhitektura}
\subfile{arhitektura}

%------------------------------------------------------------------------------------------

\section{Komponente Android aplikacije}
\subfile{komponente}

%------------------------------------------------------------------------------------------
\section{Android i STB}
\section{Android i Java}
\section{Google API}
\section{Povezivanje Android uređaja preko mreže}

\end{document}

