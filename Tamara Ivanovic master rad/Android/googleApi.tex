\documentclass[android.tex]{subfiles}
\usepackage{subfiles}
\documentclass[12pt,oneside]{memoir} 
\usepackage[latinica]{matfmaster} 
\usepackage{hyperref}
%-----------------------------------------------------------------------------------------
\usepackage{listings}
\usepackage{xcolor}

\definecolor{codegreen}{rgb}{0,0.6,0}
\definecolor{codegray}{rgb}{0.5,0.5,0.5}
\definecolor{codepurple}{rgb}{0.58,0,0.82}
\definecolor{backcolour}{rgb}{0.95,0.95,0.92}

\lstdefinestyle{mystyle}{
    backgroundcolor=\color{backcolour},   
    commentstyle=\color{codegreen},
    keywordstyle=\color{magenta},
    numberstyle=\tiny\color{codegray},
    stringstyle=\color{codepurple},
    basicstyle=\ttfamily\footnotesize,
    breakatwhitespace=false,         
    breaklines=true,                 
    captionpos=b,                    
    keepspaces=true,                 
    numbers=left,                    
    numbersep=5pt,                  
    showspaces=false,                
    showstringspaces=false,
    showtabs=false,                  
    tabsize=2
}

\lstset{style=mystyle}
%-----------------------------------------------------------------------------------------

\begin{document}

Kompanija Gugl pruža skupove pravila i protokola, odnosno \textit{Google API}-je kako bi programeri mogli da obezbede interakciju svojih aplikacija sa Gugl servisima i resursima. Zahvaljujući tome aplikacije imaju mogućnosti da pristupe podacima, funkcionalnostima i drugim resursima koje Gugl nudi. Postoji više kategorija \textit{Google API-ja} od kojih su najpoznatiji: \textit{Google Cloud API} \cite{sajt:googleCloudApis}, \textit{Google Maps API} \cite{sajt:googleMaps}, \textit{YouTube API} \cite{sajt:youtubeApis} i \textit{Google Ads API} \cite{sajt:googleAds}. Spisak svih dostupnih \textit{API}-ja sa detaljnijm opisima i uslovima korišćenja mogu se pronaći na vebu \cite{sajt:googleApiSpisak}. Svaki \textit{API} služi za pružanje pristupa specifičnim funkcionalnostima i resursima kompanije. Kako bi se koristili potrebno je registrovati se na njihovom sajtu, a zatim i dobiti \textit{API} ključ koji služi za autentifikaciju i autorizaciju zahteva. Zavisno od \textit{API}-ja, neki zahtevi mogu biti besplatni, dok drugi mogu imati troškove u zavisnosti od količine korišćenja.  

\textit{Google Cloud API}-ji omogućavaju interakciju sa servisima \textit{Google Cloud Platform} odnosno funkcionalnostima računarstva u oblaku, uključujući \textit{Google Cloud Storage}, \textit{Google Cloud Functions}, \textit{Google Compute Engine} i mnoge druge. Jedan specifičan servis je \textit{Speech-to-Text API}, odnosno prevođenje govora u tekst. Servis pomoću neuronskih mreža i mašinskog učenja pretvara audio zapis u tekst. Podržava više od 120 jezika. Za sve dostupne jezike podržani su osnovni model prepoznavanja i model prepoznavanja komandi i pretrage (eng. \textit{command and search model}). Neki jezici imaju podršku i za još neke modele kao što su poboljšani poziv (eng. \textit{enhanced audio call}) i poboljšani video (eng. \textit{enhanced video}). Primeri upotrebe su prepoznavanje komandi u realnom vremenu, generisanje titlova i transkripcija audio zapisa. Preko konzole (eng. \textit{Google Cloud Console}) moguće je kreirati nove \textit{API} ključeve, upravljati načinom plaćanja usluga, pratiti protok korišćenja omogućenih servisa, upravljati podešavanjim kreiranih projekata i još mnogo toga. Kako bi se neki servis koristio potrebno je izvršiti sledeće korake: 
\begin{enumerate}
    \item Kreirati projekat u konzoli ili odabrati neki postojeći.
    \item Omogućiti željeni \textit{API} u konzoli ukoliko već nije omogućen. 
    \item Kreirati servisni račun u delu \textit{Credentials} koji se koristi za autentifikaciju kada aplikacija komunicira sa \textit{Google} servisima i generisati privatni ključ koji se u \textit{JSON} formatu čuva na uređaju.
    \item Instalirati na računaru \textit{Google Cloud SDK} koji omogućava upravljanje resursima na \textit{Google Cloud}-u putem komandne linije.
    \item Aktivirati servis pokretanjem naredne komande iz komandne linije:
    
    \verb|gcloud auth activate-service-account --key-file=PUTANJA_DO_KLJUCA|
\end{enumerate}
Nakon ovih podešavanja moguće je koristiti odabrani \textit{API} u svojoj aplikaciji.

Izgled grafičkog interfejsa  konzole, kao i gde se nalaze prethodno navedena podešavanja u konzoli se mogu videti na vebu \cite{sajt:googleConsole}.

Upotreba \textit{Google Cloud API}-ja može biti malo složenija u kombinaciji sa programskim jezikom \textit{Java} u poređenju sa nekim drugim jezicima. Programski jezik \textit{Java} zahteva dodatne korake za generisanje klijentskih biblioteka korišćenjem \textit{gRPC} (eng. \textit{Google Remote Procedure Call}) \cite{sajt:grpc} i \textit{Protobuf} (skraćeno od eng. \textit{Protocol Buffer}) datoteka \cite{sajt:protobufDev}. O ovome će biti više reči u narednom poglavlju.

Kako bi aplikacija mogla da koristi metode iz \textit{API}-ja potrebno je ostvariti konekciju sa serverom. U klasi koja služi da omogući komunikaciju između aplikacije i servera kreira se instanca klase \textit{ManagedChannel} koja je deo \textit{Java gRPC} biblioteke. Ovoj instanci se prosleđuju adresa i port \textit{gRPC} servera, kao i privatan ključ raspakovan iz JSON datoteke. Moguće je dodati i presretače (eng. \textit{interceptor}) koji pre svakog zahteva mogu da modifikuju zahtev ili odgovor. Za potrebe autentifikacije svakog zahteva putem privatnog ključa koji je prethodno generisan koristi se objekat klase \textit{ClientAuthInterceptor}. Ovaj objekat obezbeđuje da se uz svaki zahtev prema serveru pošalju sve potrebne informacije za autentifikaciju kako bi zahtev mogao da bude obrađen. 

\end{document}