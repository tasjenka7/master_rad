\documentclass[TamaraIvanovicMasterRad.tex]{subfiles}
\usepackage{subfiles}
\documentclass[12pt,oneside]{memoir} 
\usepackage[latinica]{matfmaster} 

% ------------------
\usepackage{listings}
\usepackage{xcolor}

\definecolor{codegreen}{rgb}{0,0.6,0}
\definecolor{codegray}{rgb}{0.5,0.5,0.5}
\definecolor{codepurple}{rgb}{0.58,0,0.82}
\definecolor{backcolour}{rgb}{0.95,0.95,0.92}

\lstdefinestyle{mystyle}{
    backgroundcolor=\color{backcolour},   
    commentstyle=\color{codegreen},
    keywordstyle=\color{magenta},
    numberstyle=\tiny\color{codegray},
    stringstyle=\color{codepurple},
    basicstyle=\ttfamily\footnotesize,
    breakatwhitespace=false,         
    breaklines=true,                 
    captionpos=b,                    
    keepspaces=true,                 
    numbers=left,                    
    numbersep=5pt,                  
    showspaces=false,                
    showstringspaces=false,
    showtabs=false,                  
    tabsize=2
}

\lstset{style=mystyle}
% ------------------
\begin{document}
Ovaj rad je prikazao sve korake potrebne da se implementira Android aplikacija sa funkcionalnostima daljinskog upravljača. Svi koraci koji su potrebni za povezivanje, a zatim i način upravljanja sa uređajem su intuitivni za korisnika i poboljšavaju korisničko iskustvo.

Prilikom implementacije nailazila sam na izazove koji su odužili proces implementacije aplikacije. Glavni izazov je bilo to što je kompanija \textit{Google} prestala da daje mogućnost da se iz Srbije privatni korisnici registruju za njihove usluge. Tada je bilo potrebno naći način da se preko podataka neke kompanije kreira nalog ili da se menja smer rada. Iskoristila sam podatke porodične firme i na taj način uspela da nastavim u istom smeru. Drugi bitan izazov je bila integracija \textit{Google Speech-to-Text API}-ja. Zbog izbora programskog jezika Java za implementaciju aplikacije bilo je potrebno prevoditi protokol bafer koji generiše nove datoteke. \textit{Android Studio}-ju je često bilo potrebno više ponovnih pokretanja kako bi detektovao da su izgenerisane nove datoteke i dopustio da se koriste. 

Oba načina prepoznavanja glasovnih komandi pružaju očekivane rezulate. Bolje se pokazao način pomoću \textit{Google Speech-to-Text API}-ja zato što pruža i alternativne verzije prepoznatog, bolje reaguje na srpski akcenat pri pričanju engleskog jezika i nudi mogućnost prepoznavanja srpskog jezika. Standardni način omogućava programeru brže rešenje sa strane implementacije, bez ikakvih plaćanja usluga i nudi svoje grafičko rešenje dok osluškuje. 

Potrebe tržišta se stalno menjaju jer se menjaju zahtevi koji korisnici imaju. Samim tim i ova aplikacija ima mesta za unapređenje. Prva stvar koja bi ubrzala izvršavanje i olakšala implementaciju je prelazak na programski jezik Kotlin i koriščenje svih prednosti koje ovaj jezik pruža. Testiranje je ukazalo na potrebu za refaktorisanjem koda i poželjno poštovanje čiste arhitekture prilikom implementacije. Prepoznavanje glasovnih komandi bi trebalo proširiti na skup svih mogućih funkcionalnosti koje klijentska aplikacija pruža, a skup podržanih fraza proširiti nakon anketiranja korisnika kako bi odgovaralo što više njihovim potrebama. Kako je rad usmeren na STB uređaje jedne firme značajno poboljšanje bi bila mogućnost povezivanja sa bilo kojim STB uređajem u okolini. Kako bi se ovo implementiralo potrebno je znati naziv servisa koji uređaj pruža i koji protokol za komunikaciju koristi što nije moguće saznati bez direktnog kontakta sa pružaocima usluga.




\end{document}