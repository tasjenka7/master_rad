\documentclass[12pt,oneside]{memoir} 
\usepackage[latinica]{matfmaster} 
\usepackage{subfiles}
\usepackage{listings}
\usepackage{subcaption}


\bib{literatura-rad}


%----------------- TODO: Prebaciti u matfmaster.sty ako je moguce 
\usepackage{listings}
\usepackage{xcolor}
\usepackage{listingsutf8}

\definecolor{codegreen}{rgb}{0,0.6,0}
\definecolor{codegray}{rgb}{0.5,0.5,0.5}
\definecolor{codepurple}{rgb}{0.58,0,0.82}
\definecolor{backcolour}{rgb}{0.95,0.95,0.92}

\lstdefinestyle{mystyle}{
    backgroundcolor=\color{backcolour},   
    commentstyle=\color{codegreen},
    keywordstyle=\color{magenta},
    numberstyle=\tiny\color{codegray},
    stringstyle=\color{codepurple},
    basicstyle=\ttfamily\footnotesize,
    breakatwhitespace=false,         
    breaklines=true,                 
    captionpos=b,                    
    keepspaces=true,                 
    numbers=left,                    
    numbersep=5pt,                  
    showspaces=false,                
    showstringspaces=false,
    showtabs=false,                  
    tabsize=2
}


\usepackage{rotating}


\lstset{style=mystyle,
breaklines=true, 
 literate={\_}{\_}{1}
}
%------------------

\autor{Tamara D. Ivanović}
% Naslov teze na srpskom jeziku (u odabranom pismu)
\naslov{Implementacija upravljača za digitalnu televiziju korišćenjem platforme Android}

\godina{2023}

\mentor{dr Milena \textsc{Vujošević Janičić}, vanredni profesor\\ Univerzitet u Beogradu, Matematički fakultet}

% Ime i afilijacija prvog člana komisije (u odabranom pismu)
\komisijaA{dr Filip \textsc{Marić}, redovni profesor\\ Univerzitet u Beogradu, Matematički fakultet}
% Ime i afilijacija drugog člana komisije (u odabranom pismu)
\komisijaB{dr Aleksandar \textsc{Kartelj}, docent\\ Univerzitet u Beogradu, Matematički fakultet}

% Datum odbrane (odkomentarisati narednu liniju i upisati datum odbrane ako je poznat)
% \datumodbrane{}

% Apstrakt na srpskom jeziku (u odabranom pismu)
\apstr{
Digitalna televizija predstavlja oblast koja konstantno napreduje zbog potrebe da se korisnicima pruži što kvalitetnije korisničko iskustvo. Uređaji koji pružaju usluge prikazivanja digitalnog televizijskog sadžaja imaju instaliran operativni sistem Android. Cilj ovog rada je da kroz implementaciju aplikacije \textit{Daljinski za digitalnu televiziju} prikaže osnovne koncepte biblioteka za kreiranje Android aplikacija, način pretrage uređaja na lokalnoj mreži i povezivanja sa istim, kao i upravljanje uređajem za digitalnu televiziju preko mobilnog uređaja. Centralni deo implementacije je fokusiran na integraciju zadavanja glasovnih komandi sa ciljem olakšanja upotrebe uređaja. Rad prikazuje i zadavanje komandi na dva načina i upoređuje dobijene razultate.
}

% Ključne reči na srpskom jeziku (u odabranom pismu)
\kljucnereci{razvoj softvera, android, Google API, digitalna televizija}

\begin{document}
% ==============================================================================
% Uvodni deo teze
\frontmatter
% ==============================================================================

\naslovna

\komisija
%Strana sa posvetom
\posveta{Mami}
% ==============================================================================
\chapter*{Zahvalnica}
Zahvaljujem se mentorki svog rada, dr Mileni Vujošević Janičić, na svim sugestijama, pravovremenim komentarima, strpljenju i podršci kada su bili najpotrebniji. Hvala joj što se sa mnom radovala svakoj pređenoj prepreci.

Zahvaljujem se naučno-istraživačkom institutu RT-RK na potrebnim resursima i prilici da radim sa najboljim timom. Posebno hvala Ivoni, Martini, Maidu i Dušanu što su imali strpljenja za moj proces učenja i bili mi podrška na tom putu.

Zahvaljujem se Nikoli Vraniću na ustupljenim uređajima kada sam verovala da je sve propalo. Veliko hvala Aleksandri što nas je povezala.

Zahvaljujem se porodici i prijateljima na podršci. Hvala mami i tati što su podržali moj obrazovni put iako je bilo dugačak, uz njihovu ljubav i podršku sve je bilo lakše.
Hvala Petru što je uvek bio moj najveći navijač i moj glas razuma. Hvala mojim prijateljima što su imali razumevanja za manjak vremena.

% ==============================================================================

\apstrakt

\tableofcontents*

% ==============================================================================
% Glavni deo teze
\mainmatter
% ==============================================================================

% ------------------------------------------------------------------------------
\chapter{Uvod}
\subfile{uvod}
% ------------------------------------------------------------------------------


% ------------------------------------------------------------------------------
\chapter{Operativni sistem Android}
\label{sec:android}
\subfile{Android/android}

% ------------------------------------------------------------------------------
\chapter{Implementacija aplikacije}
\label{sec:implementacija}
\subfile{Implementacija/implementacija}
% ------------------------------------------------------------------------------

% ------------------------------------------------------------------------------
\chapter{Zaključak}
\subfile{zakljucak}
% ------------------------------------------------------------------------------

% ------------------------------------------------------------------------------
% Literatura
% ------------------------------------------------------------------------------
\literatura

% ==============================================================================
% Završni deo teze i prilozi
\backmatter
% ==============================================================================

% ------------------------------------------------------------------------------
% Biografija kandidata
\begin{biografija}
  \textbf{Tamara Ivanović} je rođena 7. novembra 1995. godine u Beogradu. Završila je prirodno-matematički smer u XV Beogradskoj gimnaziji 2014. godine kao nosilac Vukove diplome. Iste godine upisuje Matematički fakultet Univerziteta u Begoradu, smer Informatika. Zvanje diplomirani Informatičar stiče 2020. godine, nakon čega upisuje master studije na istom fakultetu. 
  
  U martu 2022. godine se priključuje stipendijskom programu u Naučno-istraživačkom institutu RT-RK tokom kog je imala priliku da uz stručnu pomoć razvija projekat koji je deo njenog master rada. Od novembra 2022. godine zasniva radni odnos u okviru instituta gde radi kao softverski inženjer na razvoju klijentskih aplikacija za digitalnu televiziju korišćenjem programskog jezika Kotlin. 
\end{biografija}
% ------------------------------------------------------------------------------

\end{document}
