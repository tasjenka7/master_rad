\documentclass[12pt,oneside]{memoir} 
\usepackage[latinica]{matfmaster} 
\usepackage{subfiles}
\usepackage{listings}
\usepackage{subcaption}


\bib{literatura-rad}


%----------------- TODO: Prebaciti u matfmaster.sty ako je moguce 
\usepackage{listings}
\usepackage{xcolor}
\usepackage{listingsutf8}

\definecolor{codegreen}{rgb}{0,0.6,0}
\definecolor{codegray}{rgb}{0.5,0.5,0.5}
\definecolor{codepurple}{rgb}{0.58,0,0.82}
\definecolor{backcolour}{rgb}{0.95,0.95,0.92}

\lstdefinestyle{mystyle}{
    backgroundcolor=\color{backcolour},   
    commentstyle=\color{codegreen},
    keywordstyle=\color{magenta},
    numberstyle=\tiny\color{codegray},
    stringstyle=\color{codepurple},
    basicstyle=\ttfamily\footnotesize,
    breakatwhitespace=false,         
    breaklines=true,                 
    captionpos=b,                    
    keepspaces=true,                 
    numbers=left,                    
    numbersep=5pt,                  
    showspaces=false,                
    showstringspaces=false,
    showtabs=false,                  
    tabsize=2
}

\lstset{style=mystyle}
%------------------

\autor{Tamara D. Ivanović}
% Naslov teze na srpskom jeziku (u odabranom pismu)
\naslov{Implementacija upravljača za digitalnu televiziju korišćenjem platforme Android}

\godina{2023}

\mentor{dr Milena \textsc{Vujošević Janičić}, vanredni profesor\\ Univerzitet u Beogradu, Matematički fakultet}

% Ime i afilijacija prvog člana komisije (u odabranom pismu)
\komisijaA{dr Filip \textsc{Marić}, redovni profesor\\ Univerzitet u Beogradu, Matematički fakultet}
% Ime i afilijacija drugog člana komisije (u odabranom pismu)
\komisijaB{dr Aleksandar \textsc{Kartelj}, docent\\ Univerzitet u Beogradu, Matematički fakultet}

% Datum odbrane (odkomentarisati narednu liniju i upisati datum odbrane ako je poznat)
% \datumodbrane{}

% Apstrakt na srpskom jeziku (u odabranom pismu)
\apstr{
Digitalna televizija predstavlja oblast konstantno traži načine za svoje unapređenje. Uređaji koji pružaju usluge prikazivanja ovog sadžaja imaju instaliran operativni sistem Android. Cilj ovog rada je da kroz implementaciju aplikacije \textit{Daljinski za digitalnu televiziju} prikaže osnovne koncepte biblioteka za kreiranje Android aplikacija, način pretrage uređaja na lokalnoj mreži i povezivanja sa istim, kao i upravljanje uređajem za digitalnu televiziju preko mobilnog uređaja. Svako unapređenje je praćeno željom da se korisniku olakša upotreba uređaja, sa istim ciljem je centralni deo implementacije fokusiran na integraciju zadavanja glasovnih komandi. Biće obrađeno zadavanje komandi na dva načina pružena od strane kompanije \textit{Google} i upoređeni razultati dobijeni primenom istih.
}

% Ključne reči na srpskom jeziku (u odabranom pismu)
\kljucnereci{razvoj softvera, android, Google API, digitalna televizija}

\begin{document}
% ==============================================================================
% Uvodni deo teze
\frontmatter
% ==============================================================================

\naslovna

\komisija
%Strana sa posvetom
\posveta{Mami}

\apstrakt

\tableofcontents*

% ==============================================================================
% Glavni deo teze
\mainmatter
% ==============================================================================

% ------------------------------------------------------------------------------
\chapter{Uvod}
\subfile{uvod}
% ------------------------------------------------------------------------------


% ------------------------------------------------------------------------------
\chapter{Operativni sistem Android}
\label{sec:android}
\subfile{Android/android}

% ------------------------------------------------------------------------------
\chapter{Implementacija aplikacije}
\label{sec:implementacija}
\subfile{Implementacija/implementacija}
% ------------------------------------------------------------------------------

% ------------------------------------------------------------------------------
\chapter{Zaključak}
\subfile{zakljucak}
% ------------------------------------------------------------------------------

% ------------------------------------------------------------------------------
% Literatura
% ------------------------------------------------------------------------------
\literatura

% ==============================================================================
% Završni deo teze i prilozi
\backmatter
% ==============================================================================

% ------------------------------------------------------------------------------
% Biografija kandidata
\begin{biografija}
  \textbf{Tamara Ivanović} je rođena 7. novembra 1995. godine u Beogradu. Završila je prirodno-matematički smer u XV Beogradskoj gimnaziji 2014. godine kao nosilac Vukove diplome. Iste godine upisuje Matematički fakultet Univerziteta u Begoradu, smer Informatika. Zvanje diplomirani Informatičar stiče 2020. godine, nakon čega upisuje master studije na istom fakultetu. 
  
  U martu 2022. godine se priključuje stipendijskom programu u Naučno-istraživačkom institutu RT-RK tokom kog je imala priliku da uz stručnu pomoć razvija projekat koji je deo njenog master rada. Od novembra 2022. godine zasniva radni odnos u okviru instituta gde radi kao softverski inženjer na razvoju klijentskih aplikacija za digitalnu televiziju korišćenjem programskog jezika Kotlin. 
\end{biografija}
% ------------------------------------------------------------------------------

\end{document}
