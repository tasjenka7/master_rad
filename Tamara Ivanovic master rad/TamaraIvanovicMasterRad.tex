\documentclass[12pt,oneside]{memoir} 
\usepackage[latinica]{matfmaster} 
\usepackage{subfiles}
\usepackage{listings}
\bib{literatura-rad}


%----------------- TODO: Prebaciti u matfmaster.sty ako je moguce 
\usepackage{listings}
\usepackage{xcolor}
\usepackage{listingsutf8}

\definecolor{codegreen}{rgb}{0,0.6,0}
\definecolor{codegray}{rgb}{0.5,0.5,0.5}
\definecolor{codepurple}{rgb}{0.58,0,0.82}
\definecolor{backcolour}{rgb}{0.95,0.95,0.92}

\lstdefinestyle{mystyle}{
    backgroundcolor=\color{backcolour},   
    commentstyle=\color{codegreen},
    keywordstyle=\color{magenta},
    numberstyle=\tiny\color{codegray},
    stringstyle=\color{codepurple},
    basicstyle=\ttfamily\footnotesize,
    breakatwhitespace=false,         
    breaklines=true,                 
    captionpos=b,                    
    keepspaces=true,                 
    numbers=left,                    
    numbersep=5pt,                  
    showspaces=false,                
    showstringspaces=false,
    showtabs=false,                  
    tabsize=2
}

\lstset{style=mystyle}
%------------------

\autor{Tamara D. Ivanović}
% Naslov teze na srpskom jeziku (u odabranom pismu)
\naslov{Implementacija upravljača za digitalnu televiziju korišćenjem platforme Android}

\godina{2023}

\mentor{dr Milena \textsc{Vujošević Janičić}, vanredni profesor\\ Univerzitet u Beogradu, Matematički fakultet}

% Ime i afilijacija prvog člana komisije (u odabranom pismu)
\komisijaA{dr Filip \textsc{Marić}, redovni profesor\\ Univerzitet u Beogradu, Matematički fakultet}
% Ime i afilijacija drugog člana komisije (u odabranom pismu)
\komisijaB{dr Aleksandar \textsc{Kartelj}, docent\\ Univerzitet u Beogradu, Matematički fakultet}
% Ime i afilijacija trećeg člana komisije (opciono)
% \komisijaC{}
% Ime i afilijacija četvrtog člana komisije (opciono)
% \komisijaD{}
% Datum odbrane (odkomentarisati narednu liniju i upisati datum odbrane ako je poznat)
% \datumodbrane{}

% Apstrakt na srpskom jeziku (u odabranom pismu)
\apstr{%
 Apstrakt rada na srpskom jeziku u odabranom pismu
}

% Ključne reči na srpskom jeziku (u odabranom pismu)
\kljucnereci{analiza, geometrija, algebra, logika, računarstvo, astronomija}

\begin{document}
% ==============================================================================
% Uvodni deo teze
\frontmatter
% ==============================================================================

\naslovna

\komisija
%Strana sa posvetom
\posveta{Mami, tati i dedi}

\apstrakt

\tableofcontents*

% ==============================================================================
% Glavni deo teze
\mainmatter
% ==============================================================================

% ------------------------------------------------------------------------------
\chapter{Uvod}
% ------------------------------------------------------------------------------


% ------------------------------------------------------------------------------
\chapter{Android}
\label{chp:android}
\subfile{Android/android}

% ------------------------------------------------------------------------------
\chapter{Implementacija aplikacije}
\label{chp:implementacija}
\subfile{Implementacija/implementacija}
% ------------------------------------------------------------------------------


% ------------------------------------------------------------------------------
\chapter{Zaključak}
% ------------------------------------------------------------------------------

% ------------------------------------------------------------------------------
% Literatura
% ------------------------------------------------------------------------------
\literatura

% ==============================================================================
% Završni deo teze i prilozi
\backmatter
% ==============================================================================

% ------------------------------------------------------------------------------
% Biografija kandidata
\begin{biografija}
  \textbf{Vuk Stefanović Karadžić} (\emph{Tršić,
    26. oktobar/6. novembar 1787. — Beč, 7. februar 1864.}) bio je
  srpski filolog, reformator srpskog jezika, sakupljač narodnih
  umotvorina i pisac prvog rečnika srpskog jezika.  Vuk je
  najznačajnija ličnost srpske književnosti prve polovine XIX
  veka. Stekao je i nekoliko počasnih mastera.  Učestvovao je u
  Prvom srpskom ustanku kao pisar i činovnik u Negotinskoj krajini, a
  nakon sloma ustanka preselio se u Beč, 1813. godine. Tu je upoznao
  Jerneja Kopitara, cenzora slovenskih knjiga, na čiji je podsticaj
  krenuo u prikupljanje srpskih narodnih pesama, reformu ćirilice i
  borbu za uvođenje narodnog jezika u srpsku književnost. Vukovim
  reformama u srpski jezik je uveden fonetski pravopis, a srpski jezik
  je potisnuo slavenosrpski jezik koji je u to vreme bio jezik
  obrazovanih ljudi. Tako se kao najvažnije godine Vukove reforme
  ističu 1818., 1836., 1839., 1847. i 1852.
\end{biografija}
% ------------------------------------------------------------------------------

\end{document}
