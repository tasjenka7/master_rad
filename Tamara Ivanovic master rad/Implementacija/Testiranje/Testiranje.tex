\documentclass[../TamaraIvanovicMasterRad.tex]{subfiles}
\usepackage{subfiles}
\documentclass[12pt,oneside]{memoir} 
\usepackage[latinica]{matfmaster} 

% ------------------
\usepackage{listings}
\usepackage{xcolor}

\definecolor{codegreen}{rgb}{0,0.6,0}
\definecolor{codegray}{rgb}{0.5,0.5,0.5}
\definecolor{codepurple}{rgb}{0.58,0,0.82}
\definecolor{backcolour}{rgb}{0.95,0.95,0.92}

\lstdefinestyle{mystyle}{
    backgroundcolor=\color{backcolour},   
    commentstyle=\color{codegreen},
    keywordstyle=\color{magenta},
    numberstyle=\tiny\color{codegray},
    stringstyle=\color{codepurple},
    basicstyle=\ttfamily\footnotesize,
    breakatwhitespace=false,         
    breaklines=true,                 
    captionpos=b,                    
    keepspaces=true,                 
    numbers=left,                    
    numbersep=5pt,                  
    showspaces=false,                
    showstringspaces=false,
    showtabs=false,                  
    tabsize=2
}

\lstset{style=mystyle}
% ------------------
\begin{document}

Bitan deo razvoja softvera čini testiranje. \cite{book:testiranjeRigorousSoftwareDev} Uz pomoć njega se proveravaju kvalitet i funkcionalnost softvera što olakšava uočavanje i ispravljanje grešaka na vreme. Testiranje pomaže i da se proveri da li su svi zahtevi koji su postavljeni pri planiranju ispunjeni. 

Jedan od oblika testiranja koji se najčešće primenjuje je testiranje jedinica koda (eng. \textit{unit testing}). Kod ovog testiranja testiraju se manje komponente aplikacije, kao na primer metode, i proverava se da li rade kako je očekivano. Moguće je kreirati lažne (eng. \textit{mock}) vrednosti da simuliraju komunikaciju sa drugim klasama, mrežom, bazom podataka i sl. jer direktna međusobna komunikacija u ovim testovima nije dozvoljena. Pored automatizovanih testova u koje spada testiranje jedinica koda, veliku primenu u razvoju softvera ima i manuelno testiranje. 

Za testiranje jedinica koda aplikacija napisanih u programskom jeziku Java najčešče se koristi radni okvir \textit{JUnit} \cite{sajt:junit}. Pored ovog radnog okvira za testiranje aplikacije \textit{Daljinski za digitalnu televiziju} korišćen je i radni okvir \textit{Mockito} \cite{sajt:mockito} koji pruža dobru simulaciju lažnih vrednosti. Izvršeni su testovi jedinica koda za pet klasa ove aplikacije, ukupno 38 jediničnih testova. Napisani su jedinični testovi za: dohvatanje instance aplikacije, proveru hosta, porta, naziva servisa i proveru zatvaranja, otvaranja i slanja podataka preko soketa. Pored ovih testova testirane su sve funkcije iz klasa \textit{CommandsHandler} i \textit{StreamingRecognizeClient} gde je provereno da li svaki klik poziva adekvatnu funkciju i da li se svaka fraza iz skupa podržanih poziva ispravno.

Android pruža mogućnost kreiranja testova koji omogućavaju da se testiraju delovi koda koji imaju interakciju sa grafičkim interfejsom aplikacije \cite{sajt:instrumentedT}. Ovi testovi se izvršavaju pomoću \textit{AndroidX Test} biblioteke koja obezbeđuje veliki broj API-ja koje se mogu koristiti za testiranje. Za potrebe aplikacije \textit{Daljinski za digitalnu televiziju} korišćen je \textit{Espresso} \cite{sajt:espresspT}. Na ovaj način testirane su funkcionalnosti za izbor STB uređaja sa kojim će se korisnik povezati. 

Testiranje celokupne aplikacije je izvršeno i manuelno sa većim fokusom na funkcionalnosti nakon povezivanja. Ovo testiranje je izvršeno više puta tokom implementacije aplikacije kako bi se dobili korisni komentari od strane korisnika. Nekoliko kolega je učestvovalo u testiranju i pružilo sugestije i impresije o radu aplikacije. Korišćenje svih dugmića za daljinski upravljač se pokazalo 100\% uspešnim. Problemi su se javljali prilikom korišćenja prepoznavanja glasa. Nijedna fraza nije pokazala apsolutnu tačnost, kraće fraze kao što su \textit{ok}, \textit{back}, \textit{up}, \textit{down}, \textit{next} i \textit{silent} su imale bolju prolaznost od dužih fraza i reči koje imaju više načina izgovora. Standardni način prepoznavanja govora je pokazao situacije u kojima ili ne prepozna šta je rečeno ili prepozna pogrešno. \textit{Google} način je dovodio do situacija gde prepozna deo fraze, pogotovo ako se napravi kraća pauza između reči, kao i do situacija da uopšte ne osluškuje. 


\end{document}