\documentclass[../TamaraIvanovicMasterRad.tex]{subfiles}
\usepackage{subfiles}
\documentclass[12pt,oneside]{memoir} 
\usepackage[latinica]{matfmaster} 

% ------------------
\usepackage{listings}
\usepackage{xcolor}

\definecolor{codegreen}{rgb}{0,0.6,0}
\definecolor{codegray}{rgb}{0.5,0.5,0.5}
\definecolor{codepurple}{rgb}{0.58,0,0.82}
\definecolor{backcolour}{rgb}{0.95,0.95,0.92}

\lstdefinestyle{mystyle}{
    backgroundcolor=\color{backcolour},   
    commentstyle=\color{codegreen},
    keywordstyle=\color{magenta},
    numberstyle=\tiny\color{codegray},
    stringstyle=\color{codepurple},
    basicstyle=\ttfamily\footnotesize,
    breakatwhitespace=false,         
    breaklines=true,                 
    captionpos=b,                    
    keepspaces=true,                 
    numbers=left,                    
    numbersep=5pt,                  
    showspaces=false,                
    showstringspaces=false,
    showtabs=false,                  
    tabsize=2
}

\lstset{style=mystyle}
% ------------------
\begin{document}

% -----------------------------------------------------------------------
U ovom poglavlju će biti opisana konkretna implementacija aplikacije \textit{Daljinski za digitalnu televiziju}. Prvo će biti navedene i objašnjene biblioteke i softveri koje je potrebno instalirati. Nakon toga će biti opisan rad aplikacije koji uključuje način instalacije, pokretanje i korišćenje aplikacije. Zatim će biti opisana struktura projekta sa detaljnim opisom koda.
% -----------------------------------------------------------------------

\section{Potrebne instalacije}

\section{Opis rada aplikacije}
\subfile{opis_rada}
% Opis pokretanja aplikacije, kako izgledaju ekrani u kom koraku



\section{Struktura projekta}
\subfile{struktura}

% Opis podele koda, osnovno o fajlovima i onda fajl po fajl opisi bitnijih delova, sta je korisceno, kako, zasto
\end{document}