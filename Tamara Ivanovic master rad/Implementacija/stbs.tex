\documentclass[struktura.tex]{subfiles}
\usepackage{subfiles}
\documentclass[12pt,oneside]{memoir} 
\usepackage[latinica]{matfmaster} 

\begin{document}
Brojne aktivnosti su potrebne da se izvrše pre nego što korisnik može da počne da koristi daljinsko upravljanje uređajem za digitalnu televiziju. Provera povezanosti na internet, potraga za uređajima na mreži koji poseduju zadatu aplikaciju kao i uparivanje sa njima čine jednu logičku celinu. Klase koje omogućavaju sve navedeno grupisane su u paket \textbf{stbs}.

\textbf{ChooseAdapter} klasa nasleđuje klasu \textit{RecyclerView.Adapter} i služi da prikaže listu stb uređaja. Obezbeđuje čuvar pogleda (eng. \textit{ViewHolder}) za svaku stavku iz liste i povezuje njegove podatke za odgovarajući pogled iz \textit{ViewHolder}-a. Klasa je kreirana tako da se koristi uz \textit{RecyclerView} koji je pogodniji za rad sa velikim skupovima dodataka u odnosu na \textit{ListView} jer omogućava lakše proširenje veličine. Konstruktor ove klase se koristi pri ubrizgavanju (eng. inflate) pogleda za svaku stavku \textit{RecyclerView}-a. Metod \textit{registerClickListener} postavlja vrednost \textit{clickListener}-u koji će omogućiti da pri pritisku stavke iz \textit{RecyclerView}-a pozove klasu \textit{OnItemClickedListener}. S obzirom da klasa nasleđuje \textit{RecyclerView.Adapter} klasu nekoliko njenih metoda je  prepisano(eng. \textit{override}). \textit{onCreateViewHolder()} se poziva pri kreiranju \textit{ViewHolder}-a, \textit{onBindViewHolder()} kada je potrebno da se prispoje podaci za \textit{ViewHolder} i \textit{getItemCount()} vraća ukupan broj stavki odnosno uređaja. Još jedna bitna metoda je \textit{refresh()} koja ažurira podatke prikazane u \textit{RecyclerView}-u. Mehanizam po kom radi je da ažurira privatni skup \textit{stbs} sa podacima koji su prosleđeni i poziva metod \textit{notifyDataSetChanged()} koji ažurira poglede u \textit{RecyclerView}-u. \textit{ChooseAdapter} klasa poseduje unutrašnju (eng. \textit{inner}) klasu \textbf{ChooseViewHolder}. To je čuvar pogleda za stavke iz \textit{RecyclerView}-a. Obezbeđuje metodu \textit{getNameView()} koja vraća tekstualno polje tipa \textit{TextView} koji skladišti ime uređaja.

\textbf{ChooseConnection} je klasa koja predstavlja početnu i glavnu aktivnost aplikacije. Odgovorna je za prikaz liste uređaja koji su otkriveni na mreži i omogućuje korisniku da se poveže sa nekim od tih uređaja. U metodu \textit{onCreate()} koji se poziva pri kreiranju aktivnosti postavlja se plan ekrana, osluškivač klika (eng. \textit{click listener}) za unos koda, osluškivač klika za adapter za povezivanje i proverava se da li je potrebno tražiti dopuštenje za korišćenje mikrofona, kao i da li je već dopušteno. Nakon svega toga se poziva i metod \textit{loadDevices()}. Tokom 15 sekundi koliko odbrojava tajmer prikazan je dijalog progresa (eng. \textit{Progress Dialog}) sa porukom "Pretraga uređaja" i čeka se da metod \textit{getStbList} iz klase \textit{DiscoveryHandler} vrati listu uređaja. Osluškivač klika za adapter za povezivanje reaguje kada korisnik odabere uređaj. Tada se poziva \verb|new ConnectToStb().execute(position);| pri čemu se kreira instanca klase ConnectToStb koja proširuje AsyncTask interfejs. U pozadini se izvršava povezivanje sa klijentom preko UDP protokola, proverava se da li je konekcija sačuvana ili ne. U slučaju da konekcija nije sačuvana otvoriće se polje za unos koda, pritiskom dugmeta će odreagovati osluškivač klika za unos koda. Tom prilikom će biti pozvan metod \textit{saveServer()} koji će sačuvati konekciju. Još jedan važan metod je \textit{openRemoteView()} koji kreira nameru prema klasi RemoteView \verb|Intent intent = new Intent(ChooseConnection.this, RemoteView.class);| i započinje aktivnost sa tom namerom \verb|startActivity(intent);|

% je aktivnost koja je odgovorna za prikaz liste stb-ova koji su otkriveni na lokalnoj mreži i omogućuje korisniku da se poveže na neki od njih. Uključuje RecyclerView koji prikazuje listu dostupnih stb-ova i feature <<naci prevod>> za uparivanje Android uređaja sa odabranim stb-om. Ovaj feature prikazuje prozor sa poljem za unos koda za uparivanje i dugmetom za slanje unetog koda. Kada se odabere stb i uređaj upari sa njim otvara se RemoteView aktivnost. Kako bi se obezbedilo da korisnik ne mora da bira uređaj svaki put kada otvori aplikaciju obezbeđeno je da se u SharedPreferences čuva adresa odabranog stb-a. 

\textbf{DiscoveryHandler}

\textbf{NsdDIscover}

\textbf{Stb}


\end{document}