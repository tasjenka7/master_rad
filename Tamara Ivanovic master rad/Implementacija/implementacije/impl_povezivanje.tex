\documentclass[struktura.tex]{subfiles}
\usepackage{subfiles}
\documentclass[12pt,oneside]{memoir} 
\usepackage[latinica]{matfmaster} 

% --------
\usepackage{listings}
\usepackage{xcolor}
\usepackage{listingsutf8}

\definecolor{codegreen}{rgb}{0,0.6,0}
\definecolor{codegray}{rgb}{0.5,0.5,0.5}
\definecolor{codepurple}{rgb}{0.58,0,0.82}
\definecolor{backcolour}{rgb}{0.95,0.95,0.92}

\lstdefinestyle{mystyle}{
    backgroundcolor=\color{backcolour},   
    commentstyle=\color{codegreen},
    keywordstyle=\color{magenta},
    numberstyle=\tiny\color{codegray},
    stringstyle=\color{codepurple},
    basicstyle=\ttfamily\footnotesize,
    breakatwhitespace=false,         
    breaklines=true,                 
    captionpos=b,                    
    keepspaces=true,                 
    numbers=left,                    
    numbersep=5pt,      
    inputencoding=utf8,
    showspaces=false,                
    showstringspaces=false,
    showtabs=false,                  
    tabsize=2
}

\lstset{style=mystyle}

%----------- 
\begin{document}
Nakon uspešnog pronalaska uređaja na mreži i prikaza liste na ekranu korisnika potrebno je obezbediti da se korisnik klikom na odabrani uređaj poveže sa istim. Za izvršenje ovog zadatka prvenstveno je potrebno obezbediti klasu koja omogućava komunikaciju korišćenjem UDP protokola (eng. \textit{User Datagram Protocol}) iz razloga što servis koji se koristi za potrebe ove aplikacije poržava UDP protokol. Glavne funkcionalnosti za UDP komunikaciju su prikazane u listingu  \ref{lst:udpClient}.

\lstinputlisting[language=Java, caption= {Klasa UdpClient}, label = {lst:udpClient}]{kodovi/UdpClient.java}


Takođe, kreirana je i jedna unutrašnja klasa koja izvršava asinhroni zadatak u pozadini. \verb|doInBackground| je funkcija koja pomoću klijenta za UDP protokol povezuje sa uređajem na osnovu informacija koje prethodno dobila o njemu. U slučaju da uređaj uparen sa korisnikovim mobilnim uređajem potrebno je poslati komandu za uparivanje o čemu će biti više reči u nastavku. Implementacija ove unutrašnje klase nazvane \textit{ConnectToStb} se nalazi u listingu \ref{lst:connectToStb}.

\lstinputlisting[language=Java, caption= {Klasa ConnectToStb}, label = {lst:connectToStb}]{kodovi/ConnectToStb.java}


\end{document}