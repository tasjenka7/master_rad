\documentclass[struktura.tex]{subfiles}
\usepackage{subfiles}
\documentclass[12pt,oneside]{memoir} 
\usepackage[latinica]{matfmaster} 

% --------
\usepackage{listings}
\usepackage{xcolor}
\usepackage{listingsutf8}

\definecolor{codegreen}{rgb}{0,0.6,0}
\definecolor{codegray}{rgb}{0.5,0.5,0.5}
\definecolor{codepurple}{rgb}{0.58,0,0.82}
\definecolor{backcolour}{rgb}{0.95,0.95,0.92}

\lstdefinestyle{mystyle}{
    backgroundcolor=\color{backcolour},   
    commentstyle=\color{codegreen},
    keywordstyle=\color{magenta},
    numberstyle=\tiny\color{codegray},
    stringstyle=\color{codepurple},
    basicstyle=\ttfamily\footnotesize,
    breakatwhitespace=false,         
    breaklines=true,                 
    captionpos=b,                    
    keepspaces=true,                 
    numbers=left,                    
    numbersep=5pt,      
    inputencoding=utf8,
    showspaces=false,                
    showstringspaces=false,
    showtabs=false,                  
    tabsize=2
}

\lstset{style=mystyle}

%----------- 


\begin{document}
Za pretragu uređaja koji se nalaze na istoj mreži koristi se klase iz paketa \verb|android.net.nsd|. Otkrivanje mrežnih servisa (eng. \textit{Network Service Discovery (NSD)}) \cite{sajt:nsd} obezbeđuje klase koje pružaju usluge pronalaska svih servisa koje pružaju uređaji na lokalnoj mreži. Potrebno je kreirati jednu klasu u kojoj će se koristiti mogućnosti za otkrivanje koje pruža ova biblioteka i koja je nazvana \textit{NsdDiscover}. Pored ove klase potrebno je napraviti klasu \textit{DiscoveryHandler} koja koristi prethodno kreirane metode u svrhu upravljanja pretragom i rezultatima.

Unutar klase \textit{NsdDiscover} se inicijalizuju osluškivač pretrage (eng. \textit{discovery listener}) i osluškivač rezultata (eng. \textit{resolve listener}). Oba predstavljaju instance klasa iz menadžera za otkrivanje mrežnih servisa (eng. \textit{NsdManager}) i imaju svoje predefinisane metode koje je potrebno prepisati. U listingu \ref{lst:nsd_listeners} su prikazani koraci za inicijalizaciju ova dva osluškivača. 

\lstinputlisting[language=Java, caption= {Inicijalizacija osluškivača pretrage i rezultata}, label = {lst:nsd_listeners}]{kodovi/NsdListeners.java}

Potrebno je definisati i jednu funkciju povratnog poziva (eng. \textit{callback}) \verb|public void onDiscover(NsdServiceInfo service){}| koja će biti prepisana u klasi \textit{DiscoveryHandler} prilikom kreiranja instance klase \textit{NsdDiscover}. Klasa \textit{DiscoveryHandler} je kreirana da bi pružila logiku koja će se izvršavati za pretragu. Ovo je prikazano u listingu \ref{lst:discovery_handler} putem definicije funckije koja kreira listu \textbf{stb} uredjaja \verb|getStbList|.

\lstinputlisting[language=Java, caption= {Metod klase DiscoveryHandler za pretragu uređaja}, label = {lst:discovery_handler}]{kodovi/DiscoveryHandler.java}

\end{document}