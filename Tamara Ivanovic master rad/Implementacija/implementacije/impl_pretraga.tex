\documentclass[struktura.tex]{subfiles}
\usepackage{subfiles}
\documentclass[12pt,oneside]{memoir} 
\usepackage[latinica]{matfmaster} 

% --------
\usepackage{listings}
\usepackage{xcolor}
\usepackage{listingsutf8}

\definecolor{codegreen}{rgb}{0,0.6,0}
\definecolor{codegray}{rgb}{0.5,0.5,0.5}
\definecolor{codepurple}{rgb}{0.58,0,0.82}
\definecolor{backcolour}{rgb}{0.95,0.95,0.92}

\lstdefinestyle{mystyle}{
    backgroundcolor=\color{backcolour},   
    commentstyle=\color{codegreen},
    keywordstyle=\color{magenta},
    numberstyle=\tiny\color{codegray},
    stringstyle=\color{codepurple},
    basicstyle=\ttfamily\footnotesize,
    breakatwhitespace=false,         
    breaklines=true,                 
    captionpos=b,                    
    keepspaces=true,                 
    numbers=left,                    
    numbersep=5pt,      
    inputencoding=utf8,
    showspaces=false,                
    showstringspaces=false,
    showtabs=false,                  
    tabsize=2
}

\lstset{style=mystyle}

%----------- 


\begin{document}
Sa ciljem pretrage uređaja na istoj mreži u skupu paketa koje su dostupne nalazi se paket  \verb|android.net.nsd|. Otkrivanje mrežnih servisa (eng. \textit{Network Service Discovery, skraćeno NSD}) \cite{sajt:nsd} obezbeđuje klase koje pružaju usluge pronalaska svih servisa koje pružaju uređaji koji su povezani na lokalnu mrežu. Za potrebe aplikacije kreirana je klasa \textit{NsdDiscover} koja koristiti mogućnosti za prepoznavanje koje pruža ova biblioteka. Pored ove klase kreirana je klasa \textit{DiscoveryHandler} koja koristi prethodno kreirane metode u svrhu upravljanja pretragom i rezultatima.

Da bi se iskoristio NSD aplikacija mora da implementira osluškivač pretrage (eng. \textit{discovery listener}) i osluškivač rezultata (eng. \textit{resolve listener}). Ovi osluškivači reaguju na otkriće servisa na mreži, kao i na gubitak servisa sa mreže. NSD menažer (eng. \textit{NsdManager}) je klasa omogućava implementaciju ovih osluškivača i obezbeđuje proces identifikacije uređaja i servisa na lokalnoj mreži. Za implementaciju NSD-a su u klasi \textit{NsdDiscover} inicijalizovani osluškivači i prepisane njihove predefinisane metode da odgovaraju potrebama aplikacije.

Osluškivač pretrage reaguje na različite događaje u procesu otkrivanja servisa. Metode koje su dostupne da se prepišu i prilagode potrebama aplikacije su:
\begin{itemize}
    \item \textit{onDiscoveryStarted} --- automatski se poziva kada počne pretraga, uz pomoć klase \textit{WiFiManager} dohvataju se dostupne informacije o bežičnoj internet konekciji,
    \item \textit{onServiceFound} --- poziva se kada se otkrije neki servis na lokalnoj mreži, ovde se vrše provere da li servis koji se traži i definiše se šta se izvršava ukoliko je pronađen traženi servis,
    \item \textit{onServiceLost} --- poziva se kada servis više nije dostupan,
    \item \textit{onDiscoveryStopped} --- poziva se kada se zaustavi pretraga, 
    \item \textit{onStartDiscoveryFailed} --- poziva se ukoliko nakon nekog vremena nije bilo moguće da se pokrene pretraga,
    \item \textit{onStopDiscoveryFailed} --- poziva se ukoliko nakon nekog vremena nije bilo moguće da se zaustavi pretraga.
\end{itemize}

Osluškivač rezultata je zadužen da kada se pronađu servisi da dohvati i upravlja informacijama koje su dobijene. Neke od informacija koje se dobijaju ovim putem su naziv hosta i broj porta. Metode koje je moguće prepisati prilikom inicijalizacije osluškivača rezultata su:
\begin{itemize}
    \item \textit{onServiceResolved} --- ukoliko su pronalazak servisa i dohvatanje informacija o njemu uspešno izvršeni poziva se ova metoda koja dalje prosleđuje informacije,
    \item \textit{onResolveFailed} --- definiše ponašanje ukoliko rešavanje ne uspe i pruža informacije o grešci koja je nastupila.
\end{itemize}

Koriščenje osluškivača na ovaj način obezbeđuje da se asinhrono izvršava otkrivanja i rešavanje mrežnih servisa. Ovim se sprečava da dođe do blokiranja korisničkog interfejsa dok se izvršavaju ove operacije. Samim tim korisnik nije svestan posla koji se dešava u pozadini i obezbeđuje se bolje korisničko iskustvo. 

Potrebno je definisati i jednu funkciju povratnog poziva (eng. \textit{callback}) \verb|public void onDiscover(NsdServiceInfo service){}| koja će biti pozivana kada se pozove metoda \textit{onServiceResolved} i kojoj će biti prosleđene informacije koje su dobijene.  U klasi \textit{DiscoveryHandler} prilikom kreiranja instance klase \textit{NsdDiscover} će ova funkcija povratnog poziva biti prepisana. Klasa \textit{DiscoveryHandler} je kreirana da bi pružila logiku koja će se izvršavati za pretragu. Ovo je prikazano u listingu \ref{lst:discovery_handler} putem definicije funckije koja kreira listu STB uređaja \verb|getStbList|.

\lstinputlisting[language=Java, caption= {Metod klase DiscoveryHandler za pretragu uređaja}, label = {lst:discovery_handler}]{kodovi/DiscoveryHandler.java}

\end{document}