\documentclass[struktura.tex]{subfiles}
\usepackage{subfiles}
\documentclass[12pt,oneside]{memoir} 
\usepackage[latinica]{matfmaster} 

% --------
\usepackage{listings}
\usepackage{xcolor}
\usepackage{listingsutf8}

\definecolor{codegreen}{rgb}{0,0.6,0}
\definecolor{codegray}{rgb}{0.5,0.5,0.5}
\definecolor{codepurple}{rgb}{0.58,0,0.82}
\definecolor{backcolour}{rgb}{0.95,0.95,0.92}

\lstdefinestyle{mystyle}{
    backgroundcolor=\color{backcolour},   
    commentstyle=\color{codegreen},
    keywordstyle=\color{magenta},
    numberstyle=\tiny\color{codegray},
    stringstyle=\color{codepurple},
    basicstyle=\ttfamily\footnotesize,
    breakatwhitespace=false,         
    breaklines=true,                 
    captionpos=b,                    
    keepspaces=true,                 
    numbers=left,                    
    numbersep=5pt,      
    inputencoding=utf8,
    showspaces=false,                
    showstringspaces=false,
    showtabs=false,                  
    tabsize=2
}

\lstset{style=mystyle}

%----------- 


\begin{document}
Komunikacija aplikacije sa STB uređajem je implementirana putem UDP protokola kao što je prikazano u prethodnom poglavlju. Za slanje komandi implementirana je funkcija \verb|send(String message)| koja kreira novu nit u kojoj bajtove poruke pakuje u jedan \textit{DatagramPacket}\cite{sajt:datagram} i preko soketa šalje povezanom uređaju. Preduslov da ova funckija radi je da je soket povezan, odnosno da je komunikacija ostvarena za zadatu mrežu i port. Implementacija funkcije je zadata u listingu \ref{lst:send}.

\lstinputlisting[language=Java, caption= {Funkcija za slanje komandi preko UDP protokola}, label = {lst:send}]{kodovi/SendCommand.java}

Poruka koja se šalje predstavlja konstantu za klik dugmeta na daljinskom uprvaljaču. U Androidu događaj tastature (eng. \textit{Key Event}) je događaj kada korisnik pritisne taster na tastaturi ili nekom od podržanih uređaja. Svaki od tastera ima svoju numeričku vrednost, odnosno k\^{o}d tastature (eng.\textit{ Key Code}). Oni zajedno olakšavaju interakciju sa uređajima koji koriste neki od podržanih uređaja. Nije potrebno znati vrednosti ovih konstanti napamet jer im se može pristupiti uključivanjem paketa \verb|android.view.KeyEvent|. 

Može se napraviti klasa koja upravlja komandama sa imenom \textit{CommandsHandler}. I u njoj implementirati funkciju za svako dugme koje je moguće pritisnuti. Kada se u glavnoj klasi pritisne dugme na primer za odlazak na početni ekran potrebno je pozvati funkciju na sledeći način: 

\verb|Singleton.getInstance().getCommandsHandler().onHomeClicked();|

S obzirom da je kreirana Singleton instanca aplikacije potrebno je dohvatiti tu instancu i nad njom instancu klase koja upravlja komandama i zatim i funkciju koja obrađuje traženi zahtev. Linija koda koja se izvršava pri ovom pozivu iz klase \textit{CommandsHandler} je 

\verb|getClient().send(String.valueOf(android.view.KeyEventKEYCODE_HOME));|


\end{document}