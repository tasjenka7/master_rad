\documentclass[struktura.tex]{subfiles}
\usepackage{subfiles}
\documentclass[12pt,oneside]{memoir} 
\usepackage[latinica]{matfmaster} 

% --------
\usepackage{listings}
\usepackage{xcolor}
\usepackage{listingsutf8}

\definecolor{codegreen}{rgb}{0,0.6,0}
\definecolor{codegray}{rgb}{0.5,0.5,0.5}
\definecolor{codepurple}{rgb}{0.58,0,0.82}
\definecolor{backcolour}{rgb}{0.95,0.95,0.92}

\lstdefinestyle{mystyle}{
    backgroundcolor=\color{backcolour},   
    commentstyle=\color{codegreen},
    keywordstyle=\color{magenta},
    numberstyle=\tiny\color{codegray},
    stringstyle=\color{codepurple},
    basicstyle=\ttfamily\footnotesize,
    breakatwhitespace=false,         
    breaklines=true,                 
    captionpos=b,                    
    keepspaces=true,                 
    numbers=left,                    
    numbersep=5pt,      
    inputencoding=utf8,
    showspaces=false,                
    showstringspaces=false,
    showtabs=false,                  
    tabsize=2
}

\lstset{style=mystyle}

%----------- 
\begin{document}

\subsection{Standardni način implementacije prepoznavanja govora}
Kao standardni način za prepoznavanje glasovnih komandi koji je obezbeđen od strane \textit{Google}-a se smatra upotreba klase \textit{Recognizer Intent}. Ova klasa je deo \textit{Speech Recognizer API}-ja ugrađenog u Android, a sve metode koje su definisane u njemu je potrebno izvršavati na glavnoj niti (eng. \textit{Main Thread}). Da bi se pokrenuo proces prepooznavanja govora kreira se namera sa akcijom \verb|RecognizerIntent.ACTION_RECOGNIZE_SPEECH|. Ova naredba pokreće aktivnost koja sluša korisnikov govor i prepoznaje ga. \textit{Recognizer Intent} pruža korisne opcije kojima se može precizirati kako sistem za prepoznavanje govora treba da se ponaša i kako komunicira sa korisnikom. Neke od opcija su:
\begin{enumerate}
    \item \textbf{EXTRA\_LANGUAGE\_MODEL} koja se koristi za odabir modela jezika za prepoznavanje govora. Jedan primer je \textbf{LANGUAGE\_MODEL\_FREE\_FORM} koji se preporučuje za prepoznavanje slobodnog stila govora.
    \item \textbf{EXTRA\_PROMPT} koja omogućava definisanje poruke koja će se prikazati korisniku prilikom slušanja.
    \item \textbf{EXTRA\_MAX\_RESULTS} koja omogućava ograničavanje maksimalnog broja rezulata koje će vratiti.
\end{enumerate}

Namera koja je kreirana se koristi u paru sa klasom \textit{ActivityResultLauncher}. Instanca se registruje u kodu pozivanjem metode \textit{registerForActivityResult} koja kao argumente prima \textit{ActivityResultContract} koji definiše ulazne i izlazne tipove i funkciju povratnog poziva koja prima izlaz. Pozivanjem metoda \textit{launch} sa argumentom definisane namere se pokreće pretraga i po završetku aktivira funkcija povratnog poziva koja obrađuje rezultat.

\subsection{Prepoznavanje govora pomoću Google računarstva u oblaku}


\subsection{Poređenje dostupnih rešenja}

\end{document}