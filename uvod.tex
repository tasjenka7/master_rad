\documentclass[TamaraIvanovicMasterRad.tex]{subfiles}
\usepackage{subfiles}
\documentclass[12pt,oneside]{memoir} 
\usepackage[latinica]{matfmaster} 

% ------------------
\usepackage{listings}
\usepackage{xcolor}

\definecolor{codegreen}{rgb}{0,0.6,0}
\definecolor{codegray}{rgb}{0.5,0.5,0.5}
\definecolor{codepurple}{rgb}{0.58,0,0.82}
\definecolor{backcolour}{rgb}{0.95,0.95,0.92}

\lstdefinestyle{mystyle}{
    backgroundcolor=\color{backcolour},   
    commentstyle=\color{codegreen},
    keywordstyle=\color{magenta},
    numberstyle=\tiny\color{codegray},
    stringstyle=\color{codepurple},
    basicstyle=\ttfamily\footnotesize,
    breakatwhitespace=false,         
    breaklines=true,                 
    captionpos=b,                    
    keepspaces=true,                 
    numbers=left,                    
    numbersep=5pt,                  
    showspaces=false,                
    showstringspaces=false,
    showtabs=false,                  
    tabsize=2
}

\lstset{style=mystyle}
% ------------------
\begin{document}
Unapređenje i konstantni razvoj Interneta sa sobom je dovelo do napretka u mnogim oblastima među kojima je i digitalna televizija. Ova oblast se konstantno razvija, a samim tim i uređaji koji se koriste za gledanje digitalne televizije preko interneta (eng. \textit{Set-Top Box}, skraćeno STB). STB uređaji imaju integrisan operativni sistem Android što pruža mogućnost da rade skoro sve što i svaki Android uređaj. Zahvaljujući Androidu moguće je povezati se sa ovim uređajem putem mobilnih uređaja i zadavati mu komande. U ovom radu biće implementirana aplikacija \textit{Daljinski za digitalnu televiziu} koja ima za cilj da prući korisniku što bolje korisničko iskustvo. Pored povezivanja na željeni uređaj i zadavanja komandi kao na regularnom daljinskom upravljaču biće moguće zadavati glasovne komande. Za ovu funkcionalnost je potrebno obezbediti pretvaranje audio zapisa u tekst. Kompanija Gugl (eng. \textit{Google}) pruža interfejs za programiranje aplikacija (eng. \textit{application programming interface}, skraćeno \textit{API}) koji se zove \textit{Speech-to-Text} i koji omogućava korišćenje servisa koji uz pomoć naprednih tehnologija vešačke inteligencije precizno konvertuju govor u tekst. U paraleli sa ovim načinom koji pruža i podršku za srpski jezik biće implementiran i standardni način koji je podržan za prepoznavanje govora kako bi se uporedili.

U poglavlju 2 opisuju se istorijat i arhitektura Androida. Zatim se uvode glavne komponente svake Android aplikacije. Poglavlja 2.4. i 2.5. govore na koji način je Android povezan sa STB uređajima i kako je programski jezik Java pronašao svoje mesto u implementaciji Android aplikacija. Na samom kraju poglavlja dat je osvrt na mogućnosti koje pruža upotreba \textit{Google API} i na koji način se koriste.

Poglavlje 3 opisuje implementaciju aplikacije \textit{Daljinski za digitalnu televiziju}. Polazi se od instalacija koje je potrebno izvršiti da bi bilo moguće implementirati aplikaciju. Dat je opis rada aplikacije korak po korak, sa opisom svih mogućnosti koje aplikacija pruža. Poglavlja 3.4. i 3.5. pružaju informacije o potrebnim dozvolama, biblotekama i zavisnostima koje su neophodne da bi se aplikacija kreirala. Svi resursi koji su korišćeni i kreirani zajedno sa opisom najbitnijih biblioteka su navedeni u poglavlju 3.6. Nakon ovoga navedene su sve klase koje su implementirane, kako su organizovane i kroz dijagrame klasa i paketa opisane njihove međusobne povezanosti. Bitan osvrt na način implementacije glavnih funkcionalnosti aplikacije kao što su pretraga uređaja, komunikacija sa STB uređajem i zadavanje glasovnih komandi su opisani u poslednjem delu poglavlja 3. Na samom kraju ovog dela upoređeni su predloženi načini implementacije glasovnih komandi. 

Poglavlje 4. prikazuje izvšena testiranja aplikacije i do kojih rezultata su oni doveli.
U poslednjem poglavlju izveden je zaključak do kog se došlo pri izradi rada i problemi sa kojima se susretalo. Takođe, dat je i predlog izmena i unapređenja kreiranog rešenja.


\end{document}